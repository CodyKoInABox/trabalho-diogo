\documentclass{article}

% Language setting
% Replace `english' with e.g. `spanish' to change the document language
\usepackage[brazil]{babel}
\usepackage[utf8]{inputenc}

% Set page size and margins
% Replace `letterpaper' with `a4paper' for UK/EU standard size
\usepackage[a4paper,top=2cm,bottom=2cm,left=3cm,right=3cm,marginparwidth=1.75cm]{geometry}
\usepackage{url}

% Useful packages
\usepackage{amsmath}
\usepackage{graphicx}
\usepackage[colorlinks=true, allcolors=blue]{hyperref}

\title{Estratégias de gerenciamento de custos e métodos de otimização de recursos no desenvolvimento de software}
\author{Maruan Biasi El Achkar, e Victor Matheus Moreira}


\begin{document}
\maketitle


\begin{abstract}
Durante o ciclo de vida de um projeto de software, existem inúmeras oportunidades para desperdiçar recursos, porém, com conhecimento de métodos e estrategias de gerenciamento de custos, é possível otimizar a utilização desses recursos para diminuir o desperdício, isso inclui desde as fases de planejamento até a manutenção de um programa.
\end{abstract}

\section{Introdução}
Gerência de custos é a área responsável pela organização financeira de um projeto, inclui desde planejamentos e estimativas até orçamentos e controle de gastos para garantir que o objetivo seja alcançado dentro do orçamento e com o menor desperdício de recursos possível. Neste artigo, exploramos diferentes estratégias para diminuir os recursos necessários no desenvolvimento de um software utilizando métodos de otimização e fazendo escolhas mais conscientes durante o planejamento, execução e manutenção de um aplicação.

\section{Áreas de Gerenciamento de Recursos}
Por ser uma área extensa e abrangente, a gerência de recursos pode ser divida em partes de menor escopo, permitindo um foco maior nos assuntos a serem resolvidos, um de cada vez.

\subsection{Análise de escopo}
A primeira etapa de um projeto é fazer uma análise de escopo e mercado, utilizando as seguintes perguntas: existe um mercado para o meu produto? O escopo do projeto e viável para minha empresa?. Essas respostas são extremamente importantes para decidir se vale ou não a pena continuar com uma ideia. Existem diversos casos onde mesmo ideias boas com grande potencial de mercado acabam falhando, pois o escopo era grande demais para a organização que estava a desenvolvendo. Por exemplo, mesmo que exista mercado para um novo sistema de pagamentos pela internet, uma empresa pequena teria bastante dificuldade em desenvolver esse projeto, já que seu escopo é muito grande e envolve muitas áreas diferentes, desde questões financeiras até jurídicas e de segurança. Por esses motivos, é importante considerar a própria capacidade de desenvolver o que está sendo aspirado antes de iniciar os trabalhos.

\subsection{Planejamento de projeto}
Vem logo após a ideia inicial e definição de escopo. Seu foco é planejar todo o ciclo de vida do projeto e fazer as escolhas que irão definir os aspectos do sistema. Nessa fase, são criados planos de como o projeto será desenvolvido, entregue, manutenido e em alguns casos aposentado. Além disso, serão feitas decisões referentes à arquitetura da aplicação, tecnologias a serem utilizadas e equipes responsáveis por cada tarefa. Documentos serão escritos detalhando todos os processos planejados e suas respectivas datas.

\subsection{Estimativas de custo}
Nesta fase, serão estipuladas estimativas de recursos necessários para cada parte da execução do projeto. Esses dados são gerados a partir de informações como: tempo de desenvolvimento, quantidade de pessoas necessárias, ferramentas licenciadas, impostos e outros custos extras. Também é importante incluir reservas, geralmente divididas em reservas de contingência, para riscos já identificados e reservas gerenciais, para possíveis problemas extras que possam surgir. As estimativas, na maioria das vezes, não acertam exatamente o valor real que será gasto, porém, permite que a organização responsável tenha uma ideia boa do tamanho do projeto. Em casos onde as estimativas são muito altas, empresas podem fazer alterações no planejamento ou até mesmo desistir de um projeto. Essa é uma fase essencial que não pode ser ignorada, como os recursos grandes ainda não foram gastos, vale mais a pena fazer mudanças significantes ao plano de execução agora do que após o início dos trabalhos.

\subsection{Definição de orçamento}
Consiste em conseguir a aprovação dos recursos estimados para o projeto. Geralmente, orçamentos não são referentes a todo o ciclo de vida de um projeto, e sim à próxima etapa a ser realizada, isso dá mais flexibilidade aos investidores a mudarem de ideia durante o desenvolvimento do produto, caso sua execução não siga suas expectativas ou estimativas prévias. Esse também é o motivo que leva projetos desorganizados a falharem, se uma organização não alcança suas metas, ela corre o risco de perder seu orçamento futuro.

\subsection{Controle de custos}
O controle de custos começa após a liberação do orçamento e engloba todo o monitoramento necessários para garantir que recursos não sejam desperdiçados. É uma área imensamente extensa com vários artigos escritos especificamente sobre ela. A forma mais básica de controle de custos é comparar os custos reais as estimativas feitas anteriormente, isso mostra se o projeto está gastando mais ou menos do que o esperado, permitindo melhores alterações no orçamento ou na forma de trabalho para equilibrar os gastos. As entregas estipuladas no plano de projeto devem ser analisadas e documentos do tipo backlog devem ser escritos sobre o andamento das atividades. Novas previsões de custos serão feitas com base nos gastos reais e solicitações de mudanças de orçamento e plano execução podem ser emitidas. Especialistas na área podem ser contratados para ajudar com análises de variação, tendencias e reservas. Também existes diversos softwares que facilitam todo esse controle.


\section{Escolha de ferramentas}
A utilização das ferramentas certas pode consagrar ou acabar com um projeto, principalmente na área de recursos. Para este tópico, utilizaremos jogos eletrônicos como exemplo. Imagine o seguinte, uma empresa quer desenvolver um jogo 3D, para isso, precisa de um software que permite a criação de modelos tridimensionais, como carros e casas, por ser uma empresa pequena, ela decide utilizar o programa Blender, que é gratuito, porém, todos seus funcionários estão acostumados a utilizar o Maya, um programa pago. É necessário levar em consideração o valor e tempo que serão gastos em treinamento em comparação com o capital economizado por um software mais barato, também precisamos lembrar da qualidade do produto final e como esse comparada as economias, não adianta ganhar nas ferramentas, mas perder nas vendas.

\subsection{Introdução de novas ferramentas com custo fixo}
Para saber se vale a pena utilizar uma ferramenta nova de custo fixo, ou seja, com um valor único de licença por empresa ou funcionário, precisamos levar em consideração as seguintes variáveis: custo da ferramenta antiga, custo da ferramenta nova, custo do treinamento dos funcionários, tempo de adaptação, valor gasto por dia e impacto nas vendas. Para facilitar essa conta, podemos utilizar a seguinte equação:
$$ L = (Cn + Ct + Cd * Tt + Iv) - Ca $$
\textbf{L} é o lucro após a introdução da nova ferramenta, caso seja positivo, vale a pena, caso seja negativo, não vale. \\
\textbf{Cn} é o custo da ferramenta nova, valor total gasto em licenças dessa ferramenta para toda a equipe. \\
\textbf{Ct} é o custo de treinamento, o valor gasto com treinamento de todos os funcionários para saberem utilizar a nova ferramenta.
\textbf{Cd} é o custo por dia extra de desenvolvimento, uma estimativa de quanto é gasto por dia de atraso, já que se considera que o treinamento irá atrasar a execução do projeto. \\
\textbf{Tt} é o tempo de treinamento em dias, a quantidade de dias que o treinamento dura. \\
\textbf{Iv} é o impacto de vendas, pode ser positivo ou negativo e deve ser o valor financeiro a ser ganho ou perdido com o impacto dessa nova ferramenta nas vendas, será positivo caso a ferramente aumente a qualidade do produto final e negativo caso diminua a qualidade do produto final. \\
\textbf{Ca} é o custo da ferramente antiga, valor total gasto em licenças dessa ferramenta para toda a equipe.\\
\textbf{Resultado:} Quando maior o valor de L mais vale a pena mudar de ferramenta, já que essa variável simboliza o valor a ser ganho ou perdido com a introdução da nova ferramenta.


\section{Formação de equipes}
A existência de diferentes equipes em uma organização otimiza e facilita o desenvolvimento de projetos complexos, ao permitir a separação de tarefas por afinidade e habilidade. Conseguir formar boas equipes é essencial na gerência de custos, por reduzir os recursos desperdiçados. Cada time deve possuir um líder, esse deve ser alguém com conhecimento abrangente em todas as áreas do projeto, incluindo áreas que serão trabalhadas por outras equipes, isso melhora a gerência interna, além da comunicação e coordenação com outras equipes. Já os outros participantes devem ser especialistas em uma área, para usar ao máximo os benefícios da divisão de tarefas. Uma empresa com boas e especializadas equipes consegue alcançar metas de forma mais organizada, com mais velocidade e eficácia, tudo isso enquanto economiza recursos valiosos.

\section{Manutenção}
A manutenção é a fase mais longa do ciclo de vida de um produto de software, pois engloba tudo desde sua primeira venda até seu último uso. Manter um software significa fazer as atualizações necessárias para manter o programa funcionando de forma que atenda aos clientes. É importante sempre lembrar, principalmente na área de gerência, que a prioridade é atender as necessidades do cliente e não ter o melhor software possível. Tentar alcançar a perfeição é um desperdício imenso de recursos, pois boa parte do esforço nunca será reconhecido ou utilizado por nenhum dos clientes. O princípio de Pareto diz que 80\% dos resultados vem de 20\% das causas, em software, isso sugere que deve-se priorizar o desenvolvimento de funcionalidades que correspondem aos 80\%¨, e não as 20\%, principalmente se a maioria dos clientes precisa apenas do que está incluso nos 80\%.


\end{document}


fontes

Estratégias para gerenciamento de custos em projetos​
https://www.wrike.com/project-management-guide/faq/what-is-cost-management-in-project-management/

cost management
https://www.wrike.com/project-management-guide/faq/what-is-cost-management-in-project-management/

gerenciamento de custos https://www.euax.com.br/2019/02/gerenciamento-de-custos-em-projetos/

PMBOK® Guide (2021).

gerenciamento https://www.euax.com.br/2019/02/gerenciamento-de-custos-em-projetos/

unity vs unreal https://theslidefactory.com/unity-vs-unreal-cost/

80 20 rule 
https://www.forbes.com/sites/forbestechcouncil/2023/09/14/taking-a-more-strategic-approach-to-the-8020-rule-in-software/?sh=1536343f3209



\subsection{Escolha de ferramentas com custo variável}
Ferramentas de custo variável se baseiam nas vendas do produto final para definir seu valor. Cada ferramenta desse tipo tem sua própria forma de cobrança, algumas cobram por receita total e algumas por vendas unitárias, para esse artigo, usaremos a Unity e a Unreal 5 como exemplos, as duas ferramentas mais populares no desenvolvimento de jogos eletrônicos, entender os benefícios de cada uma dá uma base boa para fazer as próprias decisões em relação a outras ferramentas de custo variável. \\
Unreal: Gratuita até o produto atingir um milhão de dólares americanos em vendas, depois disso, 5\% das vendas futuras. \\
Unity: Gratuita até o produto atingir um 200.000 dólares em vendas, depois disso, cada desenvolvedor usando a ferramenta deve pagar uma licença de aproximadamente 5.000 dólares.
